\chapter{Conclusion}
\label{ch:conclusion}
This final section shows the main outcomes of the project and the conclusions that can be drawn from them.

\section{Evaluation}
\label{sec:evaluation_conclusion} 
The main aim of the project was to create an application for predicting the outcome of football matches and to assist its users on creating their own individual betting system and ultimately make a more informed bet. This was certainly achieved. The three main problems facing a user that wants to bet on football were also addressed. Firstly, the application does indeed collect and display all the relevant statistics that a football punter would want to know before making a bet and it does this in a way that is both clear and concise. Secondly, it allows its users to create their own prediction formula, which makes it unique amongst other football websites that usually are either statistics only or black-box (unmodifiable) prediction systems. Thirdly, it monitors the performance of the user which is a vital tool for improving betting decisions (or for showing a user that it might be time to give up gambling altogether).

Going into more detail with regard to the functional specifications of the application that were outlined initially, it can be seen that all of the mandatory requirements were successfully implemented. This has also been proved by the results of the System Integration Testing. Of all the initial functional requirements stated at the beginning of the project, perhaps only some of the intended data visualisations on the played match page were not fully realised due to time constraints.

The non-functional requirements laid out at the beginning were all met and exceeded expectations. 

User acceptance testing proved invaluable. Not only for getting fresh eyes on the project that might catch something missed due to over-familiarity of the project, but also for building confidence that the project was a worthwhile endeavour.

Overall, the project has been highly successful, with a very well designed web application as a result.

\section{Improvements and Future Work}
\label{sec:improvements_conclusion}
At the moment, the developed application is a prototype suggesting what the system is capable of, rather than a fully-functional and thoroughly tested application and there are many ways in which SureThing could be developed in the future.

The most logical improvement would be to support more football leagues. From the technical point of view, there is nothing in the code that could prevent the application from being extended to more than one league. The EPL was chosen because the API for it was free and to have more leagues supported would mean paying money to an API provider. As it is possible to bet on almost every match from almost every league in the world, it makes sense that SureThing would have the option of predicting the outcome of any and all of these matches. In fact, for it to be commercially viable then it must have this option as punters would want the widest possible choice of matches to bet on.

There are smaller improvements brought to light by the User Acceptance Testing that could also be made: a tutorial or help menu would be useful for those not overly familiar with football statistics or sports betting in general, and maybe a rethink of the dashboard feature.

There would also need to be a decision made about the future purpose of the application, namely whether to link it more closely to the gambling industry or to keep it separate and even turn it into an alternative to gambling.  It would be possible to affiliate the application to a specific online bookmaker, and to offer links and odds from that particular website in return for a fee. This is done by many different statistics and prediction websites. However, if the focus was to be on the user's needs, then it would be a much more useful feature to be able to compare all the odds from all the online bookmakers and recommend the best odds available. As the main aim of the project was to help users with their betting decisions, this would be the logical way forward. More prediction modules would also be of benefit to punters, many of whom probably have different ideas on what to look for when predicting a result. Things like how a team did in the match immediately beforehand, whether there has been a change in team manager or if a team has a lot of injured players can all affect the outcome of a match. Even the type of weather can suit some teams more than others. It would be best not to overwhelm the screen with statistics but having other modules available, perhaps as optional settings. 

Another possibility is to develop the project as a self-contained game acting as a substitute to the gambling experience. Even though betting on football is popular, it is obviously not as popular as football itself and the gambling elements of this application may put off some fans that just want a fun game that they can play against their friends. For this to feasible, the social media type of features would have to be greatly expanded. Signing in via Twitter or Facebook is almost compulsory these days and would be done here too. The benefits of making it even easier to sign up are evident. The way the user predictions are tracked and displayed would also need to be expanded. The point of the game would be to predict the correct outcome of selected matches and the performance of each user would be recorded on a league table. At the end of the season, the user with the most points for correct predictions could get a prize (obviously depending on the application being monetised). Perhaps smaller prizes can be allocated for the top predictor of the week or month. As well as an overall league table it would be a good idea to provide means for the creation of separate leagues, so users can compete against their friends in a private competition.

There really is a vast amount of potential in this project, with two distinct paths to look into, or even a combination of them both. It is already a useful application but with further development it could be something quite special.

\section{Personal Statement}
\label{sec:personalstatement_conclusion}
I started researching this project in September 2014 and soon after began designing and implementing the application. It would have been nice to be able to focus on this alone but instead I had to fit working on this around studies, exams and courseworks while trying my best not to ignore my husband and 1 year old daughter. Thankfully, by starting early I gave myself the best chance of producing a worthwhile project whilst still being able to fulfill my other commitments.

With regard to the development of my skills, I am most happy with how well I learned the Python programming language. I started with zero knowledge or experience of using it and now I feel I would be able to use it at a professional level. The project also helped me to realise the importance of the Test Driven Development. With the sheer amount of complicated business logic running in the background, good test coverage of this crucial functionality was essential for a smooth developmental process. 

With more time I believe this application could be taken much further, it has surprised me with the scope and scale that could be achieved with the proper development. As it stands, it is a well built, useful application and I am immensely proud of my work. 